\section{应用层}
\subsection{应用层概述}
是网络体系结构的最高层,直接为用户的应用进程提供服务。应用层协议定义了应用进程之间通信的规则和约定。
\subsection{常见应用层协议}
\subsubsection{DNS}
使用UDP, 端口号53。是一种将域名转换为IP地址的服务。域名转换为IP的过程叫做域名解析
\paragraph{域名服务器分类}
\begin{itemize}
	\item 根域名服务器:管理顶级域名服务器的信息。
	\item 顶级域名服务器:管理二级域名服务器的信息。
	\item 二级域名服务器:管理具体域名的信息。
	\item 本地域名服务器:为本地用户提供域名解析服务。
\end{itemize}

\paragraph{域名解析过程}
\begin{figure}[h]
	\centering
	\includegraphics[width=\textwidth]{images/域名解析过程.png}
	\caption{DNS解析过程示意图}
\end{figure}
\newpage
\subsubsection{电子邮件}
\begin{itemize}
	\item SMTP(Simple Mail Transfer Protocol):用于发送电子邮件的协议,工作在应用层,基于 TCP 协议,默认端口号为 25。
	\item POP3(Post Office Protocol version 3):用于从邮件服务器接收电子邮件的协议,工作在应用层,基于 TCP 协议,默认端口号为 110。
	\item IMAP(Internet Message Access Protocol):用于从邮件服务器接收和管理电子邮件的协议,工作在应用层,基于 TCP 协议,默认端口号为 143。
	\item MIME(Multipurpose Internet Mail Extensions):用于扩展电子邮件格式以支持多种媒体类型的标准。
\end{itemize}

\subsubsection{HTTP}
\paragraph{概述}
\begin{itemize}
	\item 超文本传输协议(英语:HyperText Transfer Protocol,缩写:HTTP)是一种用于分布式、协作式和超媒体信息系统的应用层协议。HTTP是万维网的数据通信的基础。
	\item 设计HTTP最初的目的是为了提供一种发布和接收HTML页面的方法。通过HTTP或者HTTPS协议请求的资源由统一资源标识符(Uniform Resource Identifiers,URI)来标识。
	\item HTTP的发展是由蒂姆·伯纳斯-李于1989年在欧洲核子研究组织(CERN)所发起。
	\item HTTP的标准制定由万维网协会(World Wide Web Consortium,W3C)和互联网工程任务组(Internet Engineering Task Force,IETF)进行协调,
	\item 最终发布了一系列的RFC,其中最著名的是1999年6月公布的 RFC 2616,定义了HTTP协议中现今广泛使用的一个版本——HTTP 1.1。
	\item 2014年12月,互联网工程任务组(IETF)的Hypertext Transfer Protocol Bis(httpbis)工作小组将HTTP/2标准提议递交至IESG进行讨论,
	\item 于2015年2月17日被批准。HTTP/2标准于2015年5月以RFC 7540正式发表,取代HTTP 1.1成为HTTP的实现标准。
\end{itemize}
\paragraph{请求方法}
\begin{itemize}
	\item GET:请求指定的页面信息,并返回实体主体。
	\item HEAD:类似于GET请求,只不过返回的响应中没有具体的内容,用于获取报头。
	\item POST:向指定资源提交数据进行处理请求(例如提交表单或者上传文件)。数据被包含在请求体中。
	\item PUT:从客户端向服务器传送的数据取代指定的文档内容。
	\item DELETE:请求服务器删除指定的页面。
	\item CONNECT:HTTP/1.1协议中预留给能够将连接改为管道方式的代理服务器。
	\item OPTIONS:允许客户端查看服务器的性能。
	\item TRACE:回显服务器收到的请求,主要用于测试或诊断。
	\item PATCH:是对PUT方法的补充,用于对已知资源进行部分修改。
\end{itemize}
\paragraph{状态码}
\begin{itemize}
	\item 1xx(信息性状态码):表示请求已被接收,继续处理。
	\item 2xx(成功状态码):表示请求已成功被服务器接收、理解、并接受。
	\item 3xx(重定向状态码):表示需要客户端采取进一步的操作以完成请求。
	\item 4xx(客户端错误状态码):表示请求包含语法错误或无法完成请求。
	\item 5xx(服务器错误状态码):表示服务器未能完成合法的请求。
\end{itemize}

\begin{center}
\small
\begin{longtable}{>{\centering\arraybackslash}p{1.6cm} >{\centering\arraybackslash}p{4cm} p{10.4cm}}
\caption{HTTP 状态码一览(常用定义)}\label{tab:http-status}\\
\toprule
状态码 & 名称 (英文) & 含义 \\
\midrule
\endfirsthead

\multicolumn{3}{c}{\tablename\ \thetable\ -- 续表} \\
\toprule
状态码 & 名称 (英文) & 含义 \\
\midrule
\endhead

\midrule \multicolumn{3}{r}{接下页续} \\
\endfoot

\bottomrule
\endlastfoot

100 & Continue & 初始的部分请求已被接收,客户端应继续发送剩余请求 \\
101 & Switching Protocols & 服务器同意切换协议(如升级到 WebSocket) \\
102 & Processing &(WebDAV)服务器已收到并在处理请求,尚无响应可返回 \\
103 & Early Hints & 在最终响应准备前提示客户端可预加载的资源 \\
\midrule
200 & OK & 请求成功并返回所请求的资源 \\
201 & Created & 请求已成功并创建了新资源 \\
202 & Accepted & 请求已接受但尚未处理(异步处理) \\
203 & Non-Authoritative Information & 返回的元信息来自本地或第三方副本,而非源服务器 \\
204 & No Content & 请求成功但无实体主体返回 \\
205 & Reset Content & 请求成功,要求客户端重置文档视图 \\
206 & Partial Content & 成功处理了范围请求,仅返回部分内容 \\
207 & Multi-Status &(WebDAV)多状态响应,返回多个独立操作的状态 \\
208 & Already Reported &(WebDAV)成员已在先前的多状态响应中报告过 \\
226 & IM Used & 服务器已完成请求,返回的内容基于实例操作(RFC 3229) \\
\midrule
300 & Multiple Choices & 多种表示可供选择 \\
301 & Moved Permanently & 资源已永久移至新 URI \\
302 & Found & 临时重定向(历史上用途多样) \\
303 & See Other & 请使用 GET 到另一个 URI 获取资源 \\
304 & Not Modified & 资源未修改,客户端可使用缓存副本 \\
305 & Use Proxy & 已废弃,指示必须通过代理访问(不推荐使用) \\
306 & (Unused) & 已废弃,历史遗留码 \\
307 & Temporary Redirect & 临时重定向,方法与主体不变 \\
308 & Permanent Redirect & 永久重定向,方法与主体不变 \\
\midrule
400 & Bad Request & 请求语法错误或参数不合法 \\
401 & Unauthorized & 未经授权,需进行身份验证 \\
402 & Payment Required & 保留(未来使用) \\
403 & Forbidden & 服务器理解请求但拒绝执行 \\
404 & Not Found & 资源不存在 \\
405 & Method Not Allowed & 请求方法不被允许 \\
406 & Not Acceptable & 无法提供客户端可接受的内容 \\
407 & Proxy Authentication Required & 需对代理进行认证 \\
408 & Request Timeout & 请求超时 \\
409 & Conflict & 请求与资源当前状态冲突 \\
410 & Gone & 资源已永久移除且无已知新地址 \\
411 & Length Required & 需要 Content-Length 头 \\
412 & Precondition Failed & 先决条件失败(If-xxx 头) \\
413 & Payload Too Large & 请求实体过大(原 Request Entity Too Large) \\
414 & URI Too Long & 请求 URI 过长(原 Request-URI Too Long) \\
415 & Unsupported Media Type & 不支持的媒体类型 \\
416 & Range Not Satisfiable & 所请求的范围无法满足 \\
417 & Expectation Failed & Expect 请求头无法满足 \\
418 & I'm a teapot &(RFC 2324)愚人节玩笑:我是茶壶,不能煮咖啡 \\
421 & Misdirected Request & 请求被错误地路由到无法响应的服务器 \\
422 & Unprocessable Entity &(WebDAV)语义错误,无法处理实体 \\
423 & Locked &(WebDAV)资源被锁定 \\
424 & Failed Dependency &(WebDAV)由于先前请求失败导致失败 \\
425 & Too Early &(RFC 8470)请求可能导致重放问题,建议稍后重试 \\
426 & Upgrade Required & 需要升级协议(例如到 TLS) \\
428 & Precondition Required & 要求先决条件以避免并发问题 \\
429 & Too Many Requests & 客户端在给定时间内发送了太多请求 \\
431 & Request Header Fields Too Large & 请求头字段过大,服务器拒绝处理 \\
451 & Unavailable For Legal Reasons & 因法律原因不可用(例如审查或屏蔽) \\
\midrule
500 & Internal Server Error & 服务器内部错误,无法完成请求 \\
501 & Not Implemented & 服务器不支持请求的功能 \\
502 & Bad Gateway & 作为网关或代理时收到上游服务器的无效响应 \\
503 & Service Unavailable & 服务器当前无法处理请求(超载或维护) \\
504 & Gateway Timeout & 作为网关或代理时未及时从上游服务器收到响应 \\
505 & HTTP Version Not Supported & 不支持的 HTTP 版本 \\
506 & Variant Also Negotiates & 内容协商导致的内部配置错误 \\
507 & Insufficient Storage &(WebDAV)服务器无法存储完成请求所需的内容 \\
508 & Loop Detected &(WebDAV)检测到无限循环 \\
510 & Not Extended & 需要扩展以响应请求 \\
511 & Network Authentication Required & 网络需要进行认证(如登录门户) \\
\end{longtable}
\end{center}

\paragraph{HTTP持续连线}
HTTP/1.0 默认每个请求/响应对使用一个独立的 TCP 连接,完成后即关闭连接。
HTTP/1.1 引入持续连接(Persistent Connections),允许在同一 TCP 连接上发送多个请求/响应对,减少连接建立和关闭的开销,提高性能。
客户端和服务器通过 Connection 头字段协商是否使用持续连接。

\paragraph{持续链接的方法}
\begin{itemize}
	\item 管道化(Pipelining):允许客户端在收到前一个响应之前发送多个请求,从而提高请求的并发性和效率。
	\item 多路复用(Multiplexing):HTTP/2 引入的技术,允许在单一连接上同时发送多个请求和响应,避免了队头阻塞问题,提高了传输效率。
	\item 服务器推送(Server Push):HTTP/2 允许服务器主动向客户端发送资源,而无需客户端明确请求,从而减少延迟。
	\item 连接管理:客户端和服务器可以通过 Connection 头字段来管理连接的生命周期,例如使用 "keep-alive" 来保持连接,或使用 "close" 来关闭连接。
\end{itemize}

\subsubsection{FTP}
\begin{itemize}
	\item 文件传输协议(英语:File Transfer Protocol,缩写:FTP)是在计算机网络的客户端和服务器间传输文件的应用层协议。
	\item 传送文件(file transfer)和访问文件(file access)之间的区别在于:前者由FTP提供,后者由NFS等应用系统提供。由RFC 959规范。
	\item FTP是8位的客户端-服务器协议,能操作任何类型的文件而不需要后续处理,就像MIME或Unicode一样,
	\item 但FTP有极高的延时,意味从开始请求到第一次接收数据间的时间非常长;并且必须不时执行一些冗长的登录进程。
\end{itemize}

\paragraph{工作模式}
\begin{itemize}
	\item 主动模式(Active Mode):客户端打开一个随机端口,向服务器的21端口发送连接请求,建立控制连接。然后,客户端监听一个随机端口,并通过控制连接向服务器发送PORT命令,告诉服务器该端口号。服务器随后从其20端口连接到客户端指定的端口,建立数据连接。
	\item 被动模式(Passive Mode):客户端打开一个随机端口,向服务器的21端口发送连接请求,建立控制连接。然后,客户端通过控制连接向服务器发送PASV命令,请求服务器打开一个随机端口进行数据传输。服务器响应并告诉客户端该端口号。随后,客户端从其随机端口连接到服务器指定的端口,建立数据连接。
	\item 区别:主动模式中服务器主动连接客户端,而被动模式中客户端主动连接服务器。被动模式更适合在防火墙或NAT环境下使用,因为客户端不需要开放端口供服务器连接。
\end{itemize}

\paragraph{常用命令}
\begin{itemize}
	\item USER:指定用户名进行登录。
	\item PASS:指定密码进行登录。
	\item LIST:列出当前目录下的文件和子目录。
	\item RETR:从服务器下载文件到客户端。
	\item STOR:将客户端的文件上传到服务器。
	\item CWD:更改当前工作目录。
	\item PWD:显示当前工作目录的路径。
	\item QUIT:退出FTP会话。
	\item TYPE:设置传输类型(ASCII或二进制)。
\end{itemize}

\paragraph{边带控制}
FTP使用两个独立的连接:控制连接和数据连接。控制连接用于传输命令和响应,而数据连接用于传输文件数据。
\begin{itemize}
	\item 控制连接(Control Connection):在客户端和服务器之间建立的持久连接,通常使用TCP的21端口。通过该连接,客户端发送FTP命令,服务器返回响应。
	\item 数据连接(Data Connection):用于传输实际的文件数据。数据连接可以是主动模式或被动模式下建立的临时连接,通常使用随机端口。
\end{itemize}

\subsubsection{P2P网络}