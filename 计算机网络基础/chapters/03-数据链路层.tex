\section{数据链路层}

\subsection{数据链路层的功能}
\subsubsection{为网络层提供的服务}
\begin{itemize}[leftmargin=*]
    \item 有确认面向连接服务
    \item 无确认无连接服务(规定时间没收到确认就重传,适用误码率高的信道,如无线通信)
    \item 有确认无连接服务(适用于实时通信,误码率低的信道,如以太网)
\end{itemize}

\subsubsection{链路管理}
链路的建立、维护、释放,信道的分配与管理。

\subsubsection{帧管理}
\begin{itemize}[leftmargin=*]
    \item 帧定界:确定帧的界限。
    \item 帧同步:分出帧的起始与终止。
    \item 透明传输:不管传输的是怎样的比特组合,都能在链路上传送。
\end{itemize}

\subsubsection{流量控制}
发送方的发送能力大于接收方的接受能力,通过某反馈机制调节(限制发送方的流量)。
\subsubsection{差错控制}
在数字通信中利用编码方法对传输中产生的差错进行控制,以提高数字消息传输的准确性。

\subsection{组帧}
\subsubsection{概念}
发送方依据规则将网络层递交的分组封装成帧,帧要加首部和尾部(分组只是帧的数据部分,所以不需要加尾部)。出错时只重发出错的帧。

\subsubsection{方法}
\begin{itemize}[leftmargin=*]
    \item 字节计数法:在帧首部放一个字节计数字段,表示帧中有多少字节。缺点:如果计数字段出错,接收方无法识别帧界限。
    \item 标志定界法:在帧的起始和结束处插入特殊的标志字节(01111110)。缺点:如果数据中出现标志字节,需要进行转义处理。
    \item 比特填充法:在数据中每出现五个连续的 1 后插入一个 0,以防止数据中出现标志序列。接收方在解码时删除这些填充的 0。
    \item 违规编码法:利用数据中某些比特组合不会出现的特性来标识帧界限。例如,在某些编码方式中,连续的 1 不会超过一定数量,可以利用这一点来定义帧的起始和结束。
\end{itemize}

\subsection{差错控制}
\subsubsection{概念}
\begin{figure}[h]
    \centering
    \includegraphics[width=\textwidth]{images/f6.png}
    \caption{差错控制概念}
\end{figure}

\begin{itemize}[leftmargin=*]
    \item 奇偶校验:在数据中添加一个奇偶校验位,使得数据中 1 的个数为奇数或偶数。接收方通过检查奇偶校验位来检测单比特错误。
    \item 循环冗余校验(CRC):对于 m 比特的帧,用生成器生成一个多项式 $G(x)$ ,计算出 $FCS$。若接收方 $G(x)$ 除无余数,则无差错。
          \[
              \begin{aligned}
                  \text{发送方(FCS)} = M(x) \cdot x^r + R(x) \\
                  \text{接收方} = M(x) \cdot x^r \bmod G(x)
              \end{aligned}
          \]
    \item 海明码:通过添加冗余位来实现错误检测和纠正。海明码可以检测并纠正单比特错误,检测双比特错误。海明码检错 d 位,需要码距为 d+1;纠错 d 位,需要码距为 2d+1。/ 首先,确定校验位(码字内从左到右编号,2 的幂位为校验位,其余的填入数据);其次,计算每个数据位影响哪几个校验位,得出校验位由哪几个数据位决定;最后,校验位的值由决定它的数据位中 1 的个数确定,偶数个 1 则校验位值填 0,奇数个填 1。
\end{itemize}

\subsection{流量控制和可靠传输}
\subsubsection{概念}
\begin{figure}[h]
    \centering
    \includegraphics[width=\textwidth]{images/f7.png}
    \caption{流量控制与可靠传输机制}
\end{figure}
注:
\begin{itemize}
    \item 通过反馈机制由接收方控制发送方,发送数据的速率。
    \item 窗口大小为 1,可保证帧的有序接收。
    \item 帧的编号是循环使用的。
    \item 数据链路层的滑动窗口协议中,窗口大小在数据传输过程中固定;传输层在数据传输中可变。
    \item 数据链路层中,流量控制机制和可靠传输机制交织在一起。
    \item 向后滑动已确认过的帧数(收到确认就向后滑动,接收缓冲区依然为接收窗口大小)。
    \item 发送方维持一组允许发送的帧序号(发送窗口)。
    \item 接收方维持一组允许接收的帧序号(接收窗口)。
    \item 接收方收到数据帧后,将窗口向前移一个位置并发回确认帧(帧落在窗口外则丢弃)。
\end{itemize}

\subsubsection{三种流量控制机制}
\begin{itemize}[leftmargin=*]
    \item 停止—等待(Stop-and-Wait):
          \[
              \text{序号} \in {0,1},\quad \text{下一个序号} = 1-\text{当前序号}.
          \]
          超时重传与最小间隔:
          \[
              t_r = t_{\text{frame}} + T_{\text{timeout}}.
          \]
    \item 后退 N 帧(Go-Back-N, GBN):
          \[
              1 \le W_t \le 2^n-1,\quad n=\text{编号位数}.
          \]
          设发送窗口为 $W_t$, 当序号 $k$ 的定时器超时,重传集合为
          \[
              {k,;k+1,\dots,k+W_t-1}\pmod{2^n}.
          \]

    \item 选择重传(Selective Repeat, SR):
          仅重传出错或超时的帧;接收端可缓存接收到的、但序号不连续的帧(只要序号在接收窗口内)。
          窗口约束为
          \[
              W_{\text{send}} + W_{\text{recv}} \le 2^n.
          \]
          常用取法:
          \[
              W_{\text{send}}=W_{\text{recv}}\le 2^{,n-1},
          \]
          以避免新旧窗口序号重叠。
\end{itemize}

\subsubsection{信道利用率与吞吐率}
设数据长度为 $L$(比特),传输速率为 $C$(比特/秒),发送到收到确认的周期为 $T$(秒)。则发送方有效发送时间为
\[
    t_{\mathrm{send}}=\frac{L}{C}.
\]
信道利用率(信道效率)
\[
    \eta=\frac{\text{有效发送时间}}{\text{发送周期}}=\frac{t_{\mathrm{send}}}{T}=\frac{L/C}{T}.
\]
信道吞吐率(单位时间内成功传输的数据量)
\[
    S=\eta\cdot C=\frac{L}{T}.
\]

\subsection{介质访问控制}

\subsubsection{概念}
解决当局域网中共用信道的使用产生竞争时,如何分配信道的使用权问题。

\subsubsection{常见的介质访问控制方法}

\begin{figure}[h]
    \centering
    \includegraphics[width=\textwidth]{images/f8.png}
    \caption{常见的介质访问控制方法}
\end{figure}

\subsubsection{介质访问控制相关协议详解}

\begin{itemize}[leftmargin=*]
    \item 时分多路访问(TDMA):物理信道按时间片分成若干个时间片,轮流分配给多个信号使用(每一时间片由复用的一个信号占用)。STDM 异步时分多路复用,按需分配时隙。
    \item 频分多路访问(FDMA):多路基带信号调制到不同频率的载波上,再进行叠加形成一个复合信号(总带宽分割成若干与传输单个信号带宽相同(或略宽,保护频带)的子信道,每个子信道传输一种信号)。
    \item 波分多路访问(WDMA):光的频分多路复用。一根光纤中传输多种不同波长(频率)的光信号。
    \item 码分多路访问(CDMA):用不同的编码区分各路原始信号。/ 将每比特时间分成 m 个更短的时间槽(芯片);每个站点指定一个唯一的 m 位代码或芯片序列(发送芯片序列表示 1,发送芯片序列反码表示 0);两个或多个站点同时发送时,各路数据在信道上线性相加。为信道中分离各路信号,要求站点芯片序列相交正交的。/ 小结:不同站点码片序列正交(规格化内积为 0)、在公共信道上,两个序列是叠加的(线性相加)。
    \item 纯ALOHA:任何一站点随时发送数据,在一段时间内没收到确认则认为发生冲突,等待一段时间后再发送数据,直到发送成功。
    \item 时隙ALOHA:将时间划分为若干个时隙,站点只能在时隙开始时发送数据,若在一个时隙内没收到确认则认为发生冲突,等待随机数个时隙后再发送数据。
    \item 1-坚持 CSMA:在发送数据前监听信道,若信道空闲则发送数据,若信道忙则持续监听直到信道空闲后立即发送数据。
    \item 非坚持 CSMA:在发送数据前监听信道,若信道空闲则发送数据,若信道忙则等待随机时间后重新监听信道。
    \item p-坚持 CSMA:在发送数据前监听信道,若信道空闲则发送数据,若信道忙则以概率 p 发送数据,以概率 1-p 等待随机时间后重新监听信道。
    \item CSMA/CD(有线局域网络):载波侦听多路访问/碰撞检测(适用于总线型网络,半双工)。/ 先听后发,边听边发(不同于 CSMA 之处),冲突停发,随机重发。/ 用截断二进制指数退避算法,等待一段随机时间(检测到信道忙)。/ 最小帧长:假设某个工作站检测到冲突发生,就发送碰撞信号,使冲突更加明显,使得所有工作站都能检测到总线发生冲突。然后每个想要发送数据的工作站,检测到总线为空,在发送数据之前,先发送一个数据帧(探测帧)。探测帧的长度既要求最快速的到达目的地(尽量小),又要保证探测帧的传递时间足够(发送时间大于所有工作站监测到冲突并发送碰撞的时间)使得其他工作站能够监听到。这个探测帧的长度就是以太网规定的最小帧长。结果就是,如果没有工作站发出碰撞信号打断探测帧的传输,那么就代表总线确实为空,并且没有工作站和“我”争抢总线资源。然后就可以正式发送数据。最小帧长,部分参考链接。/ 考虑如下极限的情况,主机发送的帧很小,而两台冲突主机相距很远。 在主机 A 发送的帧传输到 B 的前一刻,B 开始发送帧。这样,当 A 的帧到达 B 时,B 检测到冲突,于是发送冲突信号。假如在 B 的冲突信号传输到 A 之前,A 的帧已经发送完毕,那么 A 将检测不到冲突而误认为已发送成功。由于信号传播是有时延的,因此检测冲突也需要一定的时间。这也是为什么必须有个最小帧长的限制。/ 小结:最小帧长度在 10Mbps 中为 64B,即发送时间至少为 51.2μs(争用期即 RTT),即 512 bit 时间。发送方必须在发送结束前收到对方的冲突信号。所以发送时间长度至少为 RTT。
    \item CSMA/CA(无线局域网络):载波侦听多路访问/碰撞避免(减少碰撞发生的概率)。/ 发送过程中不需要进行冲突检测。
    \item RTS/CTS 机制:发送方发送 RTS(请求发送)帧,接收方收到后发送 CTS(允许发送)帧,其他站点收到 RTS 或 CTS 后进入网络分配等待状态。
    \item NAV(网络分配向量):指示站点在一段时间内不允许发送数据,以避免冲突。
    \item 令牌传递(Token Passing):令牌(一组特殊比特组成的帧)在各个结点间按某个固定次序交换,传递通路逻辑上必须是环。有令牌才可发数据。/ 帧在环上传送,每个站点都进行转发,目的站点(会维持一个副本)会在尾部设置响应比特,直到再传到发送站点将此帧撤掉。/ 轮询介质访问控制适合负载很高的广播信道(多点同时发送数据概率很大的信道)。
\end{itemize}

\begin{figure}[h]
    \centering
    \includegraphics[width=\textwidth]{images/f9.png}
    \caption{CSMA/CA 相关逻辑图}
\end{figure}

\subsection{局域网}

\subsubsection{概念}

\begin{figure}[h]
    \centering
    \includegraphics[width=\textwidth]{images/f10.png}
    \caption{局域网结构示意图}
\end{figure}

\subsubsection{IEEE 802.3 (局域网参考模型) 以太网}

\begin{itemize}
    \item LLC 逻辑链路控制:作用已经不大,网卡仅装 MAC 协议,没有 LLC 协议。
        \\ 与传输媒体无关,向网络层提供服务(面向连接、无确认无连接、有确认无连接)。
    \item MAC 介质访问控制子层:向上屏蔽物理层访问差异,提供统一的接口。
        \\ 负责介质访问控制和帧定界、差错检测、流量控制。
    \item 概念:
        \\ DIX Ethernet V2 标准与 IEEE 802.3 标准很小差别,因此,将 802.3 局域网简称为以太网。
        \\ 逻辑上总线型,物理上为星型;信息以广播方式发送;CSMA/CD 对总线访问控制。
    \item 传输介质 (网卡):
        \\ 网络适配器或网络接口卡 NIC。工作在数据链路层、物理层。
        \\ 传输介质间物理连接和电信号匹配,帧发送与接收,帧封装与拆封,介质访问控制、数据编码解码,数据缓存。
    \item 传输介质 (网线):
          \begin{table}[h]
               \centering
               \begin{tabular}{
                >{\centering\arraybackslash}p{2cm}
                >{\centering\arraybackslash}p{3.5cm}
                >{\centering\arraybackslash}p{3.5cm}
                >{\centering\arraybackslash}p{3.5cm}
                >{\centering\arraybackslash}p{3cm}
                }
                \toprule[1.2pt]
                参数   & 10BASE5    & 10BASE2    & 10BASE-T    & 10BASE-FL    \\
                \midrule[0.9pt]
                传输媒体 & 基带同轴电缆(粗缆) & 基带同轴电缆(细缆) & 非屏蔽双绞线(UTP) & 光纤对(850\\,nm) \\
                编码   & 曼彻斯特编码     & 曼彻斯特编码     & 曼彻斯特编码      & 曼彻斯特编码       \\
                拓扑结构 & 总线型        & 总线型        & 星形          & 点对点          \\
                最大段长 & 500\\,m     & 185\\,m     & 100\\,m      & 2000\\,m      \\
                最多结点 & 100        & 30         & 2           & 2            \\
                \bottomrule[1.2pt]
            \end{tabular}
            \caption{10BASEx 系列以太网介质与主要参数比较}
        \end{table}
\end{itemize}
