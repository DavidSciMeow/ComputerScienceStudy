\section{传输层}
\subsection{传输层功能}
\begin{itemize}
	\item 传输层位于网络层之上,应用层之下,主要负责在端系统之间提供逻辑通信。
	\item 复用(上层数据使用相同的传输层协议)
	\item 分用(同一传输层协议数据向上交付给不同协议的应用)
	\item 对收到的报文段首部、数据部分差错检测(网络层只检测首部)。
	\item 面向连接的 TCP,无连接的 UDP。
	\item 可靠不可靠:是否使用确认机制对传输的数据进行确认。
\end{itemize}

\subsection{传输层寻址(端口)}
\begin{itemize}
	\item 端口号(port number)用于标识主机上不同的应用进程。
	\item 端口号是一个16位的整数,取值范围为0~65535。
	\item 端口号分为三类:
	\begin{itemize}
		\item 知名端口(Well-known Ports):0~1023,由系统或知名服务使用。
		\item 注册端口(Registered Ports):1024~49151,由用户进程或应用程序注册使用。
		\item 动态/私有端口(Dynamic/Private Ports):49152~65535,通常由客户端临时分配使用,用于短期通信。
	\end{itemize}
	\item 传输层协议通过源端口号和目的端口号来区分不同的应用进程,实现多路复用和分用功能。
\end{itemize}
\begin{figure}[h]
	\centering
	\includegraphics[width=\textwidth]{images/端口号分类.png}
	\caption{端口号分类}
\end{figure}
\paragraph{熟知端口号表}
\begin{table}[h]
	\centering
	\begin{tabular}{@{}c c c@{}}
		\toprule
		服务 & 协议 & 端口号 \\
		\midrule
		HTTP & TCP & 80 \\
		HTTPS & TCP & 443 \\
		FTP & TCP & 20 (数据), 21 (控制) \\
		SSH & TCP & 22 \\
		Telnet & TCP & 23 \\
		SMTP & TCP & 25 \\
		DNS & UDP/TCP & 53 \\
		DHCP & UDP & 67 (服务器), 68 (客户端) \\
		POP3 & TCP & 110 \\
		IMAP & TCP & 143 \\
		SNMP & UDP & 161 \\
		LDAP & TCP/UDP & 389 \\
		RDP & TCP/UDP & 3389 \\
		\bottomrule
	\end{tabular}
	\caption{常见服务的熟知端口号}
	\label{tab:well-known-ports}
\end{table}

\newpage

\subsection{传输层协议}
\subsubsection{UDP协议}
\paragraph{概念与特点}
\begin{itemize}
	\item 用户数据报协议(User Datagram Protocol, UDP)是一种无连接的传输层协议。
	\item 只在 IP 数据包上增加两个基本服务,复用分用、差错检测。
	\item 无需建立连接。
	\item 接收方发现端口号不正确则丢弃,并由 ICMP 发送差错报告端口不可达。
\end{itemize}

\paragraph{UDP报文格式}

\begin{figure}[h]
	\centering
	\includegraphics[width=\textwidth]{images/UDP报文头部.png}
	\caption{UDP报文头部}
\end{figure}

\paragraph{UDP校验}
\begin{itemize}
	\item UDP 校验和用于检测数据在传输过程中是否发生错误。
	\item 计算方法:
	\begin{itemize}
		\item 将 UDP 头部、数据部分和伪首部(包含源 IP 地址、目的 IP 地址、协议号和 UDP 长度)按 16 位字进行分组。
		\item 将所有 16 位字进行二进制求和,若有进位则加到结果的最低位。
		\item 对求和结果取反,得到校验和。
	\end{itemize}
	\item 接收方收到 UDP 数据报后,使用相同的方法计算校验和,并与报文中的校验和进行比较。
	\item 若两者不匹配,则说明数据在传输过程中发生了错误,接收方会丢弃该数据报。
	\item UDP 校验和是可选的,但强烈建议使用以提高数据传输的可靠性。
\end{itemize}

\subsubsection{TCP协议}
\paragraph{概念}
传输控制协议(Transmission Control Protocol, TCP)是一种面向连接的、可靠的传输层协议。它提供端到端的可靠数据传输服务,确保数据按顺序到达且无差错。
\paragraph{特点}
\begin{itemize}
	\item 面向连接:在数据传输前需要建立连接,传输完成后释放连接。
	\item 可靠传输:通过序列号、确认号、重传机制确保数据可靠到达。
	\item 流量控制:使用滑动窗口机制控制数据流量,防止接收方被淹没。
	\item 拥塞控制:动态调整发送速率以适应网络拥塞状况。
	\item 全双工通信:允许双方同时发送和接收数据。
\end{itemize}
\paragraph{TCP报文格式}
\begin{itemize}
	\item 
	\begin{figure}[h]
		\centering
		\includegraphics[width=.8\textwidth]{images/TCP报文头部.png}
		\caption{TCP报文头部}
	\end{figure}
\end{itemize}
\paragraph{TCP的连接管理}
\begin{itemize}
	\item 三次握手建立连接:
	\begin{itemize}
		\item 第一次握手:客户端发送 SYN 报文段,进入 SYN\_SENT 状态。
		\item 第二次握手:服务器收到 SYN 报文段,回复 SYN-ACK 报文段,进入 SYN\_RECEIVED 状态。
		\item 第三次握手:客户端收到 SYN-ACK 报文段,回复 ACK 报文段,连接建立成功,双方进入 ESTABLISHED 状态。
	\end{itemize}
	\item 四次挥手释放连接:
	\begin{itemize}
		\item 第一次挥手:客户端发送 FIN 报文段,进入 FIN\_WAIT\_1 状态。
		\item 第二次挥手:服务器收到 FIN 报文段,回复 ACK 报文段,进入 CLOSE\_WAIT 状态;客户端收到 ACK 报文段,进入 FIN\_WAIT\_2 状态。
		\item 第三次挥手:服务器发送 FIN 报文段,进入 LAST\_ACK 状态。
		\item 第四次挥手:客户端收到 FIN 报文段,回复 ACK 报文段,进入 TIME\_WAIT 状态,等待一段时间后进入 CLOSED 状态;服务器收到 ACK 报文段,连接释放成功,进入 CLOSED 状态。
	\end{itemize}
\end{itemize}
\begin{figure}[h]
	\centering
	\includegraphics[width=\textwidth]{images/TCP连接管理.png}
	\caption{TCP连接管理}
\end{figure}

\paragraph{TCP可靠传输}
\begin{itemize}
	\item 校验(和 UDP 的校验一样)。
	\item 序号:建立在字节流上,而非报文段上。
	\item 确认:TCP 使用累计确认,发送一个报文段字节后才发回一个确认。
	\item 重传:
	\item - 超时(超时重传时间应略大于加权平均往返时间 RTTs)。
	\item - 冗余 ACK(某个报文段确认时捎带冗余 ACK 标表示之前丢失的报文段,期望收到之前丢失的)。规定当发送方收到同一个报文段 3 个冗余 ACK 时,可以认为跟在这个被确认报文段后的报文段已经丢失。这种技术被称为快速重传。
\end{itemize}

\paragraph{TCP流量控制}
\begin{itemize}
	\item 滑动窗口机制:接收方通过调整窗口大小来控制发送方的发送速率。
	\item - 是一个速度匹配服务,消除发送方使接收方缓存区溢出的可能性。
	\item - 要求 发送方(cwnd 拥塞窗口)$<=$ 接收方(rwnd 接收方窗口)。
	\item 接收方根据自己缓冲区大小确定接收窗口告诉发送方;
	\item 发送方根据当前网络拥塞程序估计确定拥塞窗口值。发送方取 cwnd、rwnd 中值小的作为发送窗口。
	\item 防止接收方缓冲区溢出,确保数据传输的可靠性。
	\item 与数据链路层比较:
	\item - 数据链路层:窗口大小不变化、两个相邻节点的流量控制;
	\item - 传输层:窗口可大小变化,端到端的流量控制。
\end{itemize}

\paragraph{TCP拥塞控制}
\begin{itemize}
	\item 概念:
	\item - 防止过多的数据注入到网络中。
	\item - 发送方窗口可等同于拥塞窗口 cwnd(接收方总是有足够大的缓存空间)。
	\item 两种组合四种算法:慢开始和拥塞避免、快重传和快恢复(对前者的改进)。
	\item 具体策略:
	\item - 当发送方检测到超时、TCP 链接建立时,用慢开始和拥塞避免。
	\item - 当发送方检测到冗余 ACK 时,采用快重传和快恢复。
	--------------
	\item 慢开始(slow start):
	\item - 先令拥塞窗口 cwnd = 1(mms 最大报文段长度),每收到一个确认后 cwnd 加倍(指数增大),直到达到慢开始门限 ssthresh 值后改用拥塞避免算法。
	\item 拥塞避免(congestion avoidance):
	\item - cwnd 每经过一个 RTT 就增加一个 MSS 大小(线性增大),当出现一次超时就令 cwnd = 1,ssthresh 为出现拥塞时 cwnd 的一半。
	\item 快重传(fast retransmit):
	\item - 发送方连续收到三个冗余 ACK 报文时,直接重传尚未收到的报文段,不必等待那个报文段设置的重传计时器超时。
	\item 快恢复(fast recovery):
	\item - 连续收到三个冗余 ACK,设置 cwnd = 出现拥塞时 cwnd 的一半,ssthresh = 出现拥塞时 cwnd 的一半。然后 cwnd 线性增大。
\end{itemize}

\begin{figure}[h]
	\centering
	\includegraphics[width=\textwidth]{images/f511.png}
	\includegraphics[width=\textwidth]{images/f512.png}
	\caption{拥塞控制}
\end{figure}
