\section{R语言概述}
\subsection{R 的起源与发展}
R 最早源自贝尔实验室的 S 语言的思想和实现,目标是为统计计算和图形展示提供一种交互式语言。1990 年代早期,Ross Ihaka 与 Robert Gentleman 在新西兰奥克兰大学基于 S 设计并实现了 R。随后 R 在全球范围内被学术界和工业界广泛采纳,形成了以 CRAN(Comprehensive R Archive Network)为核心的软件和文档分发体系。R 基金会(R Foundation)负责协调 R 的发展与发布,社区通过贡献包、文档和示例推动了生态的扩展。

\paragraph{演化}
R 从最初作为统计研究工具演变为数据科学常用语言之一:核心语言保持稳定,扩展主要通过 CRAN、Bioconductor、GitHub 等渠道提供大量领域包(统计建模、可视化、机器学习、生物信息等)。近年出现了诸如 \textnormal{data.table}、\textnormal{tidyverse}、\textnormal{sparklyr} 等重要项目,提升了 R 在大数据与工程化场景下的可用性。

\paragraph{R 的特点}
\begin{itemize}[leftmargin=*]
  \item 开源免费,跨平台(Windows、macOS、Linux)。
  \item 面向向量和矩阵的计算模型,内置丰富的统计函数与绘图功能。
  \item 强大的包生态(CRAN、Bioconductor 等)和活跃社区支持。
  \item 交互式分析与脚本化工作流并重,便于探索性分析与可重复研究。
  \item 支持扩展的 C/Fortran 接口以及与数据库、大数据平台的集成能力。
\end{itemize}

\subsection{安装与运行 R}
\paragraph{获取 R}
可从 CRAN 官网(https://cran.r-project.org)或各地镜像下载适用于不同操作系统的安装包;在部分 Linux 发行版上也可通过包管理器安装。

\paragraph{安装}
\begin{itemize}[leftmargin=*]
  \item Windows:下载官方安装程序并按提示安装。
  \item macOS:下载 \textnormal{.pkg} 安装包或使用 Homebrew(\textnormal{brew install r})。
  \item Linux:使用发行版的包管理器(如 \textnormal{apt}、\textnormal{yum})或从源代码编译安装。
\end{itemize}

\paragraph{运行 R}
\begin{itemize}[leftmargin=*]
  \item 交互式:\textnormal{R} 命令进入交互式控制台;Windows 有 RGui。
  \item 脚本运行:使用 \verb|Rscript your_script.R| 或 \textnormal{R CMD BATCH} 在批处理/生产环境中执行脚本。
  \item IDE:常用 \textnormal{RStudio}、\textnormal{VS Code}(配合 R 扩展)提供更友好的开发环境。
\end{itemize}

\subsection{安装与使用包}
\paragraph{什么是包}
包(package)是 R 的功能扩展单元,包含 R 代码、数据、文档、示例与测试。CRAN 是标准分发渠道,Bioconductor 面向生物信息学领域,GitHub 常用于开发版分发。

\paragraph{安装包}
\begin{itemize}[leftmargin=*]
  \item 从 CRAN:\textnormal{install.packages("pkgname")}。
  \item Bioconductor:使用 \textnormal{BiocManager\allowbreak::\allowbreak install("pkg")}。
  \item GitHub:使用 \textnormal{devtools\allowbreak::\allowbreak install\_github("user/repo")} 或 \textnormal{remotes\allowbreak::\allowbreak install\_github()}。
\end{itemize}

\paragraph{载入与使用}
使用 \textnormal{library(pkgname)} 或 \textnormal{require(pkgname)} 载入包;也可用 \textnormal{pkgname\allowbreak::\allowbreak function()} 直接调用包内函数,避免污染全局命名空间。

\paragraph{卸载包}
使用 \textnormal{remove.packages("pkgname")} 从库中移除包。

\paragraph{包的命名空间}
包拥有独立命名空间,导出的符号可被用户直接调用,未导出的符号仅供包内部使用。命名空间机制通过 \textnormal{NAMESPACE} 文件声明导出与导入规则,\textnormal{::} 与 \textnormal{:::} 运算符用于访问导出和非导出对象(后者不推荐常用)。理解搜索路径(\textnormal{search()})与命名冲突有助于排查加载问题。

\subsection{工作空间管理}
\paragraph{工作空间概念}
工作空间(workspace)指当前 R 会话中已创建的对象集合(变量、函数、数据框等)。常见操作包括列出对象、保存与恢复、清理及设置工作目录。

\paragraph{常用命令}
\begin{itemize}[leftmargin=*]
  \item \textnormal{ls()} 或 \textnormal{objects()}:列出当前对象。
  \item \textnormal{rm(x)}:删除对象,\textnormal{rm(list = ls())} 可清空当前会话对象。
  \item \textnormal{save(object, file = "file.RData")}、\textnormal{load("file.RData")}:持久化与恢复对象。
  \item \textnormal{save.image()}:保存整个工作空间(生成 \textnormal{.RData})。
  \item \textnormal{getwd()}、\textnormal{setwd()}:查看与设置工作目录。
  \item 推荐使用 R 项目(RStudio 的 \textnormal{.Rproj})来管理文件与会话,提升可重现性。
\end{itemize}

\subsection{集成开发环境 RStudio}
\paragraph{什么是 IDE}
集成开发环境(IDE)是将编辑器、控制台、调试器、项目管理、包与帮助浏览等功能整合在一起的工具,旨在提高开发效率与可视化操作体验。

\paragraph{RStudio 的主要功能与使用}
\begin{itemize}[leftmargin=*]
  \item 脚本编辑与执行(逐行或选中运行)、交互控制台。
  \item 环境/变量窗格、历史记录、文件与包管理面板。
  \item 可视化输出、绘图浏览与导出功能。
  \item 调试工具(断点、step 等)、项目支持(\textnormal{.Rproj})、R Markdown 与交互式文档支持。
  \item RStudio Server 可在远程服务器上提供基于浏览器的 IDE。
\end{itemize}

\subsection{帮助系统}
R 提供多层次帮助:函数级别(\textnormal{?mean} 或 \textnormal{help("mean")})、模糊搜索(\textnormal{??"regression"} 或 \textnormal{help.search()})、包文档(\textnormal{help(package = "ggplot2")})、示例(\textnormal{example(fun)})和详细教程/长文档(\textnormal{vignette("vignette-name", package = "pkg")})。另外 \textnormal{help.start()} 可启动本地 HTML 帮助索引。熟练使用帮助系统能显著加速学习与开发。

\subsection{R 语言与大数据}
\paragraph{与大数据平台的集成}
R 与大数据生态有多种结合方式:
\begin{itemize}[leftmargin=*]
  \item 数据库连接:通过 \textnormal{DBI}、\textnormal{RMySQL}、\textnormal{RPostgres} 等与关系型数据库直接交互。
  \item 分布式计算:\textnormal{sparklyr}、\textnormal{SparkR} 能将计算下推到 Apache Spark;\textnormal{RHadoop}、\textnormal{rhdfs} 等可与 Hadoop 生态集成。
  \item 内存与高性能包:\textnormal{data.table}、\textnormal{dtplyr} 在单机上处理大数据表现优异;\textnormal{bigmemory}、\textnormal{ff} 提供对超出内存数据的支持。
  \item 并行计算:\textnormal{parallel}、\textnormal{foreach}、\textnormal{future} 等包支持多核与分布式任务并行化。
\end{itemize}

\paragraph{在数据科学中的角色}
R 在数据清洗、统计建模、可视化与快速原型方面具有优势。尽管在极大规模(PB 级)数据处理上常与 Spark、数据库或 Python 互补,但 R 在建模实验、可视化展示与统计推断领域仍然是重要工具。

\subsection{小结}
本章概述了 R 的起源与演化、安装与运行、包管理、工作空间、RStudio IDE、帮助系统以及 R 与大数据的关系。后续章节将对数据类型、控制结构和函数等主题展开详细介绍。
