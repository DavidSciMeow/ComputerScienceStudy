\section{程序设计基础}
\subsection{控制流}
\subsubsection{顺序结构}
程序按照书写顺序逐行执行,这是最基本的执行模型。程序由若干语句组成,语句按顺序依次计算并产生副作用或返回值。

\subsubsection{分支结构}
用于根据条件选择执行路径。常见形式有 `if`、`if-else`、`ifelse()` 以及 `switch()`。
示例:
\begin{verbatim}
if (x > 0) {
  print("positive")
} else if (x < 0) {
  print("negative")
} else {
  print("zero")
}
\end{verbatim}

\subsubsection{循环结构}
用于重复执行代码。常见的循环有 `for`、`while` 和 `repeat`。在 R 中优先考虑向量化操作或 `apply` 系列以提高性能。
示例:
\begin{verbatim}
for (i in 1:5) {
  print(i)
}

i <- 1
while (i <= 5) {
  print(i)
  i <- i + 1
}
\end{verbatim}

\subsubsection{选择结构}
`switch()` 用于根据一个表达式的值选择多个分支,适合离散的选项分发。逻辑索引用于按条件筛选或赋值,是 R 风格的“选择”操作。

\subsection{函数设计}
\subsubsection{声明、定义与调用}
函数用 `function` 关键字声明,并通过名称调用:
\begin{verbatim}
f <- function(x, y = 1) {
  x + y
}
f(2)
\end{verbatim}

\subsubsection{返回值}
函数的返回值为最后一个求值表达式的值,或显式使用 `return()` 返回。

\subsubsection{函数的输入输出}
函数参数支持位置参数、命名参数和默认值。建议为复杂函数写明参数含义并进行输入校验。

\subsubsection{环境与范围}
R 使用词法作用域(lexical scoping)。函数内部默认访问最近的环境;使用 `<<-` 可修改外层环境变量,但应谨慎使用以避免副作用。

\subsubsection{递归函数}
递归函数在函数体内调用自身,适用于分治和递归定义的问题,但需注意终止条件与性能。示例:阶乘
\begin{verbatim}
fact <- function(n) {
  if (n <= 1) return(1)
  n * fact(n - 1)
}
\end{verbatim}

\subsection{编程规范与性能优化}
\subsubsection{使用脚本}
把相关代码组织成脚本文件(`.R`),用 `source()` 加载;项目层面建议使用包结构或 `here`/`rproj` 管理路径。

\subsubsection{编程规范}
命名应有一致性(如驼峰或下划线);写足够注释,避免“魔法数字”;把可复用逻辑封装为函数,添加文档与示例。

\subsubsection{性能优化}
优先使用向量化操作、预分配存储(如 `numeric(n)`)、适当选择数据结构(`data.table`/`tibble`),并用 `Rprof()`、`profvis` 等工具定位瓶颈。示例建议:
\begin{verbatim}
# 不推荐:在循环中逐步扩展向量
v <- c()
for (i in 1:10000) v <- c(v, i)

# 推荐:预分配
v <- integer(10000)
for (i in 1:10000) v[i] <- i
\end{verbatim}

更多进阶优化包括并行计算(`parallel`, `future`)、使用 C/C++ 接口(`Rcpp`)等。

% 结束
