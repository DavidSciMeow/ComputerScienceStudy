\section{类与对象}
\subsection{面向对象程序设计方法}
\subsubsection{结构化程序设计方法的问题}
结构化程序设计强调过程和步骤,但在处理复杂系统时容易导致代码耦合、全局状态增多、可维护性差和扩展困难。

\subsubsection{对象与类的概念}
对象是包含状态(数据)和行为(方法)的实体;类是描述对象结构和行为的模板。通过类可以封装数据和操作,隐藏实现细节。

\subsubsection{面向对象程序设计的特点}
- 封装:将数据和操作封装在对象中。
- 继承:通过基类扩展新类复用代码。
- 多态:同一接口在不同类中有不同实现。

\subsubsection{R 中类的体系}
R 支持多种 OOP 系统:非正式的 S3,形式化的 S4,以及引用语义的 Reference Class(和 R6 包提供的现代引用类)。不同系统在类定义、方法分派和数据访问语义上有所区别。

\subsection{S3 类}
\subsubsection{S3 类定义}
S3 是一种轻量的面向对象机制,基于在对象上设置 `class` 属性。无需事先声明类。

\subsubsection{创建 S3 类对象}
示例:
\begin{verbatim}
person <- list(name = "Alice", age = 30)
class(person) <- "person"
person
\end{verbatim}

\subsubsection{S3 类的泛型函数}
S3 使用基于 `UseMethod()` 的泛型函数分派。例如 `print()`、`summary()` 是泛型,按对象的 `class` 调用对应方法。

\subsubsection{定义 S3 类的方法}
通过命名约定 `generic.class` 定义方法:
\begin{verbatim}
print.person <- function(x, ...) {
	cat("Person:", x$name, "(age", x$age, ")\n")
}
\end{verbatim}

\subsubsection{编写 S3 类的泛型函数}
自定义泛型:
\begin{verbatim}
greet <- function(x, ...) UseMethod("greet")
greet.person <- function(x, ...) cat("Hello,", x$name, "!\n")
\end{verbatim}

\subsection{S4 类}
\subsubsection{S4 类的定义}
S4 是更严格的 OOP 系统,要求在创建类前显式声明类和槽(slots),并支持形式化的类检查。

\subsubsection{创建 S4 类的对象}
定义与实例化示例:
\begin{verbatim}
setClass("Person",
				 slots = list(name = "character", age = "numeric"))
p <- new("Person", name = "Bob", age = 40)
\end{verbatim}

\subsubsection{访问插槽}
使用 `@` 操作符访问 S4 对象的插槽:
\begin{verbatim}
p@name
p@age <- 41
\end{verbatim}

\subsubsection{S4 类的泛型函数}
S4 使用 `setGeneric()` 和 `setMethod()` 定义泛型和方法,更适合较大项目和包开发。
示例:
\begin{verbatim}
setGeneric("greet", function(x) standardGeneric("greet"))
setMethod("greet", "Person", function(x) cat("Hi,", x@name, "\n"))
\end{verbatim}

\subsubsection{定义 S4 类的方法}
使用 `setMethod()` 为特定类签名添加实现,支持多参数的签名匹配。

\subsection{引用类}
\subsubsection{定义引用类}
引用类(Reference Classes)提供引用语义(mutable objects),适用于需要在多个地方共享可变状态的场景。可通过 `setRefClass()` 定义。

\subsubsection{创建引用类的对象}
示例:
\begin{verbatim}
Counter <- setRefClass("Counter",
	fields = list(count = "numeric"),
	methods = list(
		increment = function() count <<- count + 1,
		get = function() count
	))
c <- Counter(count = 0)
c$increment()
c$get()
\end{verbatim}

\subsubsection{访问和修改引用对象的域}
通过 `\$` 访问字段与方法,修改字段会改变对象本身(引用语义)。

\subsubsection{引用类的方法}
方法在类定义中直接声明,使用 `self` 或直接引用字段来操作状态。

\subsection{继承}
\subsubsection{S3 类的继承}
S3 通过在 `class` 属性中指定多个类名实现继承与方法回退,例如 `c("child", "parent")`,泛型会首先查找 `child` 方法,再回退到 `parent`。

\subsubsection{S4 类的继承}
S4 使用 `contains` 在 `setClass()` 中声明继承关系:
\begin{verbatim}
setClass("Employee", contains = "Person",
				 slots = list(empId = "character"))
\end{verbatim}

\subsubsection{引用类的继承}
Reference Classes 支持通过 `contains` 参数继承其他引用类。

\subsubsection{多重继承}
S4 支持多重继承(多个 `contains`),需要注意潜在的方法冲突与槽名冲突,通常通过明确方法签名解决歧义。

% 结束
