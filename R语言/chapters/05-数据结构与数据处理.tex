\section{数据结构与数据处理}
\subsection{向量}
\subsubsection{创建向量}
使用 \textnormal{c()}、\textnormal{seq()}、\textnormal{rep()} 等创建向量:
\begin{verbatim}
x <- c(1, 2, 3)
y <- seq(1, 10, by = 2)
z <- rep(0, times = 5)
\end{verbatim}

\subsubsection{使用索引访问向量}
使用下标访问(从 1 开始):\verb|x[1]|;逻辑索引和名称索引也可用。

\subsubsection{循环补齐}
循环中逐步增长向量效率低,推荐预分配或使用向量化操作。

\subsubsection{向量比较}
对向量进行比较会返回逻辑向量,例如 `x > 0`。

\subsubsection{按条件提取元素}
使用逻辑索引提取满足条件的元素:\verb|x[x > 0]|。

\subsection{矩阵和数组}
\subsubsection{创建矩阵}
用 \textnormal{matrix()} 或 \textnormal{rbind}/\textnormal{cbind} 创建矩阵:
\begin{verbatim}
m <- matrix(1:6, nrow = 2)
\end{verbatim}

\subsubsection{线性运算}
支持矩阵乘法(`%*%`)、逐元素运算、转置(`t()`)等。

\subsubsection{使用矩阵索引}
通过 `[i, j]` 访问元素,行/列切片返回矩阵或向量。

\subsubsection{Apply 函数族}
使用 `apply()`、`rowSums()`、`colMeans()` 等对矩阵按维度操作:
\begin{verbatim}
apply(m, 1, sum) # 每行求和
\end{verbatim}

\subsubsection{多维数组}
使用 `array()` 创建多维数组,使用 `[i,j,k]` 索引访问。

\subsection{数据框}
\subsubsection{创建数据框}
使用 `data.frame()` 或 `tibble::tibble()` 创建数据框:
\begin{verbatim}
df <- data.frame(a = 1:3, b = c("x","y","z"))
\end{verbatim}

\subsubsection{访问元素}
使用 `df\$col`、`df[i, j]`、`df["col"]` 访问列和单元格;使用 `subset()` 或 `dplyr` 进行筛选。

\subsubsection{使用 SQL 语句查询数据}
可用 `sqldf` 包在 R 中对数据框执行 SQL 查询:
\begin{verbatim}
library(sqldf)
sqldf("select * from df where a > 1")
\end{verbatim}

\subsection{因子}
因子用于表示分类数据,常用于统计建模与绘图;使用 `factor()` 创建并设置水平(levels)与标签(labels)。

\subsection{列表}
列表是异质容器,可存放不同类型对象:
\begin{verbatim}
lst <- list(name = "Alice", scores = c(90, 85))
\end{verbatim}
使用 `[[ ]]` 或 `\$` 访问元素。

\subsection{数据导入与导出}
\subsubsection{数据文件的读写}
常用函数:\texttt{read.csv/read.table}、\texttt{write.csv};现代包:\texttt{readr::read\_csv}、\texttt{readr::write\_csv}。

\subsubsection{rio 包}
`rio` 提供简洁的 `import()`/`export()` 接口,自动识别文件格式,方便常见数据交换。

\subsubsection{数据编辑器}
可使用 `edit()`、`utils::View()` 或 RStudio 的数据查看器进行交互式编辑/查看。

\subsection{数据清洗}
\subsubsection{数据排序}
使用 `order()`、`dplyr::arrange()` 对数据排序。

\subsubsection{数据清洗的一般方法}
- 缺失值处理(删除、填补、插值)
- 重编码与类型转换(如将字符转因子)
- 重命名列、去重、格式化日期

\subsubsection{mice 包}
`mice` 提供多重插补方法处理缺失数据,适合较复杂的缺失模式分析。

% 结束
