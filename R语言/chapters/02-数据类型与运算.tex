\section{数据类型与运算}
\subsection{基础知识}
\paragraph{向量}
向量是 R 中最基础的数据结构之一;一个向量包含同一类型的元素(numeric、integer、logical、character)。向量支持下标访问(从 1 开始),常见创建方式有 \textnormal{c(...)}、\textnormal{seq()}、\textnormal{rep()}。

\paragraph{对象}
在 R 中几乎一切都是对象:向量、矩阵、数据框、函数等都是对象,可以赋值给变量并传递给函数。

\paragraph{函数}
函数是 R 的第一类对象,既有基本函数(如 \textnormal{sum()}, \textnormal{mean()}),也可自定义函数(\textnormal{function(x) \{ ... \}})。函数参数支持位置与命名传递,可设默认值。

\paragraph{标识符与保留字}
标识符用于命名对象,必须以字母或点开头(但不能以点后接数字),后续可包含字母、数字、点和下划线。R 有若干保留字(如 \textnormal{if}, \textnormal{else}, \textnormal{for}, \textnormal{function}, \textnormal{TRUE}, \textnormal{FALSE}, \textnormal{NA}),不能用作变量名。

\subsection{数据类型}
\paragraph{基本数据类型}
\begin{table}[ht]
  \centering
  \begin{tabularx}{\linewidth}{@{}l X@{}}
    \toprule
    类型 & 说明 \\
    \midrule
    数值型 & numeric / double \\
    整数型 & integer(例如使用 \textnormal{1L} 表示整数常量) \\
    逻辑型 & logical(\textnormal{TRUE} / \textnormal{FALSE}) \\
    字符型 & character(字符串,用双引号或单引号表示) \\
    因子 & factor(用于表示分类变量,底层用整数表示并带有标签) \\
    \bottomrule
  \end{tabularx}
\end{table}

\paragraph{变量与常量}
变量通过赋值创建(如 \textnormal{x <- 1} 或 \textnormal{x = 1}),R 没有语言层面的常量关键字,但可通过约定或包实现只读行为。

\paragraph{特殊值}
\begin{table}[ht]
  \centering
  \begin{tabularx}{\linewidth}{@{}l X@{}}
  	\toprule
    符号 & 含义 \\
    \midrule
    	\textnormal{NA} & 缺失值(missing value) \\
    	\textnormal{NaN} & 不是数(not a number),通常为 \textnormal{0/0} 的结果 \\
    	\textnormal{Inf} / \textnormal{-Inf} & 正/负无穷 \\
    	\textnormal{NULL} & 表示空对象 \\
    \bottomrule
  \end{tabularx}
\end{table}

\subsection{基本运算}
\paragraph{运算符分类}
\begin{table}[ht]
  \centering
  \begin{tabularx}{\linewidth}{@{}c l l X@{}}
    \toprule
    优先级 & 类别 & 符号 & 说明 \\
    \midrule
    1 & 幂与一元运算 & \textnormal{\^} & 幂运算 \\
    1 & 幂与一元运算 & \textnormal{+}(一元) & 一元正号 \\
    1 & 幂与一元运算 & \textnormal{-}(一元) & 一元负号 \\
    2 & 乘除与模 & \textnormal{*} & 乘法 \\
    2 & 乘除与模 & \textnormal{/} & 除法 \\
    2 & 乘除与模 & \textnormal{\%\%} & 取模(返回余数) \\
    2 & 乘除与模 & \textnormal{\%/\%} & 整数除法 \\
    3 & 加减 & \textnormal{+} & 加法 \\
    3 & 加减 & \textnormal{-} & 减法 \\
    4 & 关系运算 & \textnormal{==} & 等于 \\
    4 & 关系运算 & \textnormal{!=} & 不等于 \\
    4 & 关系运算 & \textnormal{>} & 大于 \\
    4 & 关系运算 & \textnormal{<} & 小于 \\
    4 & 关系运算 & \textnormal{>=} & 大于等于 \\
    4 & 关系运算 & \textnormal{<=} & 小于等于 \\
    5 & 序列与索引运算 & \textnormal{:} & 序列生成运算符 \\
    6 & 元素级逻辑 & \textnormal{\&} & 元素级与 \\
    6 & 元素级逻辑 & \textnormal{|} & 元素级或 \\
    7 & 短路逻辑 & \textnormal{\&\&} & 短路与 \\
    7 & 短路逻辑 & \textnormal{||} & 短路或 \\
    8 & 取反 & \textnormal{!} & 逻辑取反 \\
    9 & 赋值 & \textnormal{<-} & 左赋值 \\
    9 & 赋值 & \textnormal{=} & 赋值(注意上下文差异) \\
    9 & 赋值 & \textnormal{->} & 右赋值 \\
    9 & 赋值 & \textnormal{<<-} & 全局赋值 \\
    \bottomrule
  \end{tabularx}
\end{table}

\begin{table}[ht]
  \centering
  \begin{tabularx}{\linewidth}{@{}l X@{}}
    \toprule
    数据结构 & 说明 \\
    \midrule
    向量(vector) & 同质一维数组,最常用的数据结构。 \\
    矩阵(matrix) & 同质二维数组,可用 \textnormal{matrix()} 创建。 \\
    数据框(data.frame / tibble) & 异质表格型结构,每列可为不同类型,常用于数据分析。 \\
    列表(list) & 异质的一维容器,可包含任意对象(向量、数据框、函数等)。 \\
    因子(factor) & 用于表示分类数据,便于统计建模。 \\
    \bottomrule
  \end{tabularx}
\end{table}

\paragraph{索引与子集}
R 支持多种索引方式:数字下标(从 1 开始)、逻辑下标、名称下标以及切片。使用 \verb|[ ]|、\verb|[[ ]]| 和 \verb|$| 来访问不同层级或字段。

\paragraph{示例}
\begin{table}[ht]
  \centering
  \begin{tabularx}{\linewidth}{@{}l X@{}}
    \toprule
    操作 & 例子 \\
    \midrule
    创建向量 & \textnormal{x <- c(1, 2, 3)} \\
    创建数据框 & \textnormal{df <- data.frame(a = 1:3, b = c("x","y","z"))} \\
    子集选择 & \textnormal{df[df\$a > 1, ]}、\textnormal{lst[[1]]}、\textnormal{df\$b} \\
    \bottomrule
  \end{tabularx}
\end{table}

% 结束
